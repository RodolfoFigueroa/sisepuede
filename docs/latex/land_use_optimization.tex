\documentclass[10pt]{article}

\usepackage{amsfonts, amsmath, array, fancyhdr, float, graphicx, lipsum, multirow, url}

\title{Land Use Transitions Optimization Framework for SISEPUEDE}
\begin{document}
\maketitle


% LIST PARAMETERS
%
% 

\subsubsection*{Parameters and Indicies}

Let:
\begin{itemize}
\item
$n \in \mathbb{N}^+$ be the number of land use classes

\item
$t \in \mathbb{N}$ be the time period

\item
$x_t \in \mathbb{R}^n$ be the prevalence vector at time $t$. For convenience, this is sometimes shown as simply $x$. Furthermore, since $x$ and $q$ are combined in the objective function, prevalence vectors should be expressed as a stochastic vector, i.e., so that $\sum_i x_i = 1$ and $x_i \geq 0$. Since the area of a region is generally fixed--sea level rise can be represented through transitions to flooded states--expressing land use prevalence as a fraction is relatively straight-forward.

\item
$\hat{x}$ be the target prevalence vector. Depending on the costs $r_i$ (see below), this vector may only value include legitimate target prevalence values for some classes.
 
\item

$a(t) \in \mathbb{R}^n$ be a vector of minimum prevalence fractions allowable for each 
$u_

\item
$Q(t) \in \mathbb{R}^{n \times n}$ be the exogenously specified row-stochastic transition matrix at time $t$.

\item 
$s_{ij}$ be the negative cost applied to transition probability deviations from $i$ to $j$ (in general, $s_{ij} \leq 0$)

\item 
$r_i$ be the negative cost applied to prevalence deviations for class $i$ (in general, $r_i \leq 0$)
\end{itemize}

% VARIABLES
%
%
\subsubsection*{Variables}
Let:
\begin{itemize}
\item
$q_{ij}(t)$ be the adjusted transition matrix at time $t$. Since $q \in \mathbb{R}^{n \times n}$ a matrix, we use $q^{(j)}$ to represent column $j$ and $q_i$ to represent row $i$. 

\item
$d(x, x)$ be a distance metric on $\mathbb{R}^n$
\end{itemize}

% Problem Formulation
%
%
\subsubsection*{Problem}

\begin{equation}
\begin{array}{c}
\textbf{maximize} \sum_{i, j}{d(q_{ij}, Q_{ij})s_{ij}} + \sum_{j}{d(x_tq^{(j)} - \hat{x}_j)r_j}\\
\begin{array}{rcl}
\textbf{subject to} \sum_{i, j}{d(q_{ij}, Q_{ij})s_{ij}} + \sum_{j}{d(x_tq^{(j)} - \hat{x}_j)r_j} & \, & \\
\end{array}
\end{array}
\end{equation}

\end{document}